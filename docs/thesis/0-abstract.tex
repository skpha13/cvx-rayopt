\begin{abstractpage}

\begin{abstract}{english}

Radial image reconstruction is an ongoing challenge in the computational imaging field, with applications in medical tomography as well as artistic renderings. This bachelor’s thesis addresses the aforementioned problem by exploring optimization algorithms to reconstruct images from a set of radial projections. The main objective is to implement and analyze optimization algorithms, with particular attention to aspects such as time and memory consumption, as well as the visual quality of the results.

To illustrate and demonstrate our approach, we solve the string art problem\footnote{For more on string art, see \href{https://en.wikipedia.org/wiki/String_art}{String Art}}, which is a smaller subproblem of the general radial reconstruction problem. We will introduce a method commonly used in radial image reconstruction, applied in the novel context of string art, using a custom implementation of the Radon transform. Although string art is used as a conceptual analogy, the focus of this thesis lies in the mathematical and algorithmic methods for the broader problem of radial image reconstruction.

Thus, this thesis will contribute to the field of image processing while also offering a new perspective on the artistic applications of mathematics and programming.

\end{abstract}

\begin{abstract}{romanian}

Reconstrucția imaginii radiale este o problemă curentă în domeniul imagisticii computaționale, cu aplicații în tomografia medicală și în redări artistice. Această lucrare de licență abordează problema menționată anterior, explorând algoritmi de optimizare pentru reconstrucția imaginilor dintr-un set de proiecții radiale. Obiectivul principal este implementarea și analiza algoritmilor de optimizare, cu accent pe aspecte precum timpul și memoria de calcul, precum și calitatea rezultatului obținut.

Pentru a ilustra și demonstra abordarea noastră, vom rezolva problema string art\footnote{Pentru mai multe despre string art, vedeți \href{https://en.wikipedia.org/wiki/String_art}{String Art}}, care este o subproblemă a problemei generale de reconstrucție radială. Vom introduce o metodă utilizată frecvent în reconstrucția imaginii radiale, aplicată într-un context nou, cel al string art, folosind o implementare personalizată a transformatei Radon. Deși string art este folosită ca o analogie conceptuală, accentul acestei lucrări cade pe metodele matematice și algoritmice ale problemei generale de reconstrucție a imaginii radiale.

Astfel, lucrarea contribuie la domeniul procesării imaginilor, oferind și o perspectivă nouă asupra aplicațiilor artistice ale matematicii și programării.

\end{abstract}

\end{abstractpage}